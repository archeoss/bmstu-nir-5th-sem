\chapter{Анализ предметной области}\label{ch:--anal}

\section{Способы модификации ядра Linux}\label{sec:--anal--methods}

Прежде чем приступить к обзору и анализу существующих методов модификации ядра Linux, необходимо определить, что подразумевается под модификацией ядра Linux.

\subsection{Модификация ядра Linux}\label{subsec:--anal--methods--mod}

В данном разделе описаны основные критерии модификации ядра Linux.

\begin{itemize}
    \item[$-$] \textbf{Модификация ядра Linux} $-$ это изменение кода ядра Linux, которое не включает в себя изменение структуры ядра Linux.
    Таким образом программист волен изменять любые части ядра Linux, не затрагивая саму структуру ядра.
    Любые изменения, которые включают в себя изменение структуры ядра Linux, называются \textbf{переписыванием ядра Linux}, что выходит за рамки данного исследования.
    \item[$-$] Любой дополнительный функционал, который добавляется в ядро Linux, должен оперировать существующими структурами ядра Linux.
    Иными словами, здесь и далее, мы говорим, что новые функции должны работать на уровне ядра Linux, а не на уровне пользователя.
    \\[0.5em]
    Современные операционные системы, в частности Linux, разделяет виртуальную память на пространство пользователя и пространство ядра.
    В классическом представлении, существует 4 кольца защиты, однако их точное количество зависит от множества факторов, включая архитектуру процессора, архитектуру операционной системы и т.д. Схематично это представлено на рисунке \ref{img:rings}.
    \includeimage
    {rings} % Имя файла без расширения (файл должен быть расположен в директории inc/img/)
    {f}
    {h}
    {0.75\textwidth} % Ширина рисунка
    {Кольца защиты} % Подпись рисунка
    \newpage
    \item[$-$] Главное различие между уровнем ядра и уровнем пользователя заключается в том, что в то время пока процессы на уровне пользователя оперируют ограниченными участками памяти,
    то процессы на уровне ядра имеют доступ ко всей памяти компьютера.
\end{itemize}

\subsection{Задачи модификации ядра Linux}\label{subsec:---linux}

Конкретные задачи модификации ядра крайне обширны и во многом зависят от конкретно поставленной цели конкретного проекта.
Вычисления на уровне ядра дают некоторые преимущества, однако, в большинстве случаев, они либо не являются необходимыми, либо недостатки таких подходов нивелируют эти преимущества.
Общее правило, которое может быть сформировано для всех методов, выглядит следующим образом:
\begin{itemize}
    \item[$-$] Если вычисления на уровне ядра не являются необходимыми, то они \textbf{не} должны быть реализованы.
    \vspace{5mm}\\
    \item[$-$] Если вычисления на уровне ядра все же являются необходимыми, то следует провести тщательный анализ, соизмеримое по времени с написанием самой программы, на тему того, не существует ли других альтернатив.
    \item[$-$] Только в том случае, если вычисления на уровне ядра являются необходимыми и других альтернатив не существует, то с особой осторожностью можно приступить к написанию таких программ.
\end{itemize}

Другими словами можно сказать следующее: модификация ядра Linux должна быть последним вариантом,
когда все остальные варианты были исчерпаны.
\\
Следующий список дает примеры некоторых задач, которые \textbf{должны} быть решены на уровне ядра, однако ни в коей мере он не является полным или не лишенным исключений:

\begin{enumerate}
    \item Написание приложений, таких как драйвера устройств, у которых должен быть доступ к низкоуровневым ресурсам, которые не могут быть предоставлены другими способами (например, прерывания).
    \item Реализация алгоритмов, которые должны быть выполнены с высокой точностью по времени и/или пространству (например, мониторинг ресурсов системы или совместное использование ресурсов).
    \item Написание программ, которые должны быть доступны всем пользователям системы (например, демонов).
    \item Также следует перейти в пространство ядра, где накладные расходы, такие как смена пространств пользователь-ядро, становится неприемлемым для правильной работы написанного кода.
    Чаще всего в таких случаях мы говорим об облачных вычислениях.
\end{enumerate}

\section{Базовые понятия и термины}\label{sec:--anal--terms}
\textbf{TODO: Добавить описание терминов и базовых понятий.}\\
\textbf{TODO: А это вообще нужно?}
%\textbf{Ядро Linux} $-$ это основная часть операционной системы Linux, которая обеспечивает основные функции ОС.
%Ядро Linux представляет собой набор программ, которые обеспечивают взаимодействие между различными компонентами ОС.
%\\
%\indent\textbf{Модуль ядра} $-$ это небольшая программа, которая расширяет функциональность ядра Linux.
%\\
%\indent\textbf{Драйвер} $-$ это программа, которая обеспечивает взаимодействие между ядром Linux и аппаратным обеспечением.
%\\
%\indent\textbf{Системный вызов} $-$ это специальная функция, которая предоставляет доступ к функциям ядра Linux из пользовательского приложения.
