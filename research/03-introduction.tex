\specsection{ВВЕДЕНИЕ}
% \addcontentsline{toc}{chapter}{ВВЕДЕНИЕ}

В ходе разработки комплексных систем зачастую возникает необходимость создания мощной и надёжной системы на основе Linux.
Для таких случаев появляется необходимость настраивать системное ядро таким образом,
чтобы вся система работала более эффективно и надёжно, заточенную под конкретную задачу.
Ядро Linux хоть и является универсальным, однако возникают ситуации, когда его базовая функциональность не позволяет решить поставленную задачу.\\
Так например, сегодня различные расширения ядра обеспечивают критически важные
функции для обеспечения безопасности и эффективности в облачных системах\cite{cloud-kernel}.
Также в ядро Linux внедряются различные драйвера, которые обеспечивают работу различных устройств, подключаемых к компьютеру. \\
В связи с этим возникает вопрос о модификации ядра Linux и внедрении в него новых функций.
\vspace{0.1cm}
\\
\textbf{Цель работы} $-$ классифицировать методы модификации ядра Linux.
\\
\vspace{0.1cm}
\\
Для достижения поставленной цели требуется решить следующие задачи: %FIXME
\begin{itemize}
    \item[$-$] Провести обзор существующих методов модификации ядра Linux;
    \item[$-$] Провести анализ методов модификации ядра Linux;
    \item[$-$] Сформулировать критерии классификации методов модификации ядра;
    \item[$-$] Классифицировать существующие методы модификации ядра.
\end{itemize}
