\specsection{ВВЕДЕНИЕ}
% \addcontentsline{toc}{chapter}{ВВЕДЕНИЕ}

При разрабатывании комплексных систем возникает необходимость создания надёжной системы на основе Linux.
Для таких случаев появляется настраивается ядро системы таким образом,
чтобы система работала, будучи заточенную под конкретную задачу.
Ядро Linux хоть и проектировалось быть универсальным, однако возникают ситуации, когда базовая функциональность не решает поставленные задачи.\\

Так например, сегодня расширения ядра обеспечивают
функциональность большинства облачных систем~\cite{cloud-kernel}.
Также в ядро Linux внедряются драйвера, которые обеспечивают работу устройств, подключаемых к компьютеру.
Из-за этого возникает вопрос о модификации ядра Linux и внедрении в него новых функций.
\vspace{0.1cm}
\\
\textbf{Цель работы} $-$ классификация методов модификации ядра Linux.
\\
\vspace{0.1cm}
\\
Для достижения поставленной цели требуется решить следующие задачи: %FIXME
\begin{itemize}
    \item[$-$] Провести обзор существующих методов модификации ядра Linux;
    \item[$-$] Провести анализ методов модификации ядра Linux;
    \item[$-$] Сформулировать критерии классификации методов модификации ядра;
    \item[$-$] Классифицировать существующие методы модификации ядра.
\end{itemize}
