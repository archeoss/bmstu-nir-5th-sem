\chapter*{ВВЕДЕНИЕ}
\addcontentsline{toc}{chapter}{ВВЕДЕНИЕ}

Расширяемость ядра долгое время была желаемым свойством различных операционных систем,
так как это позволяет пользователям с различными потребностями совместно использовать
ту же кодовую базу, что и сама ОС.
Эта концепция со временем широко распространилась, и такие ОС как Linux были разработаны с возможностью
легкого расширения ядра.
\\
\indentСегодня различные расширения ядра обеспечивают критически важные
функции для обеспечения безопасности и эффективности в облачных системах.
К сожалению, безопасность ядра не является идеальной, и множество ошибок в ядре продолжают вызывать проблемы\cite{bugs-pie}\cite{bugs-version}.
\\
\vspace{0.1cm}
\\
\textbf{Цель работы} $-$ классифицировать методы модификации ядра Linux.
\\
\vspace{0.1cm}
\\
Для достижения поставленной цели требуется решить следующие задачи:
\begin{itemize}
    \item[$-$] Провести обзор существующих методов модификации ядра Linux;
    \item[$-$] Провести анализ методов модификации ядра Linux;
    \item[$-$] Сформулировать критерии сравнения методов модификации ядра;
    \item[$-$] Классифицировать существующие методы модификации ядра.
\end{itemize}
